\documentclass{article}
\usepackage{graphicx}
\usepackage{csquotes}
\usepackage[ruled,vlined]{algorithm2e}
\usepackage{booktabs}  
\usepackage{siunitx}   
\usepackage[table]{xcolor}
\usepackage{array}
\usepackage{float}



\title{Semantic Stability}

\begin{document}

\maketitle

\section{Introduction}
This projects aims at testing and visualizing the responses, and more importantly the differences between responses to slightly different, but similar prompts.

\section{Theory}
A simple base prompt $p$, and a set of variant $p^*$ prompts are chosen. Using the \textit{all-MiniLM-L6-v2} model, their embedding vectors are created. Only those variant $p^*$ prompts are selected, which surpass the \textbf{0.85} similarity (calculated using \textit{cosine similarity}).

\paragraph{Base prompt.}
\texttt{Why do humans need sleep?}

\paragraph{Prompt variants.}
\begin{itemize}
    \item What makes sleep essential for humans?
    \item How does sleep benefit the human body and mind?
    \item What role does sleep play in human health and functioning?
    \item Why is it necessary for people to sleep?
    \item In what ways is sleep crucial to human well-being?
    \item What are the reasons humans can't function without sleep?
    \item Why is getting enough sleep important for humans?
    \item What happens to the human body and brain that makes sleep a necessity?
\end{itemize}

\paragraph{Filtering prompt variants.}
To control prompt diversity, we compute the semantic similarity between each variant and its base prompt.
The procedure is as follows:

\begin{enumerate}
    \item Encode the base prompt $p$ and all variants $p^*_i$ using a SentenceTransformer model.
    \item Compute cosine similarities $s_i = \text{cos\_sim}(p, p^*_i)$ for each variant.
    \item Construct a data frame containing each variant, its similarity score, and a Boolean flag
    indicating whether it meets a minimum threshold (e.g.\ $s_i \ge 0.85$).
    \item Sort the data frame in descending order by similarity.
\end{enumerate}

\begin{table}[ht]
    \centering
    \caption{Variant prompts with their similarity scores and if they remain in the experiment.}
    \label{tab:sleep_variants}
    \rowcolors{2}{gray!10}{white} % Alternating row colors
    \begin{tabular}{c >{\raggedright\arraybackslash}p{8cm} S[table-format=1.6] c}
        \toprule
        ID & Variant & {Similarity} & Keep \\
        \midrule
        3 & Why is it necessary for people to sleep? & 0.918821 & True \\
        6 & Why is getting enough sleep important for humans? & 0.882275 & True \\
        0 & What makes sleep essential for humans? & 0.850024 & True \\
        7 & What happens to the human body and brain that makes sleep a necessity? & 0.825465 & False \\
        5 & What are the reasons humans can't function without sleep? & 0.817788 & False \\
        4 & In what ways is sleep crucial to human well-being? & 0.753535 & False \\
        1 & How does sleep benefit the human body and mind? & 0.747996 & False \\
        2 & What role does sleep play in human health and functioning? & 0.693460 & False \\
        \bottomrule
    \end{tabular}
\end{table}



To determine the base response, the base prompt is sent 10 times to our chosen LLM (\textit{gpt-5-mini}), and KMeans is applied to the responses' embeddings.

\begin{figure}[H]
    \centering
    \includegraphics[width=1\linewidth]{base_response_embs_cluster.png}
    \caption{Base Responses' Embeddings Cluster}
    \label{fig:placeholder}
\end{figure}

To choose a cluster, we calculate the cosine similarity between the cluster members and choose the cluster with the most internal similarity, which in this case is \textit{Cluster 0} with a \textbf{0.936} similarity.
The cluster's centroid is calculated, and from now on it acts as the \textit{base response embedding.}
\end{document}
